\documentclass{article}
\usepackage{xeCJK}
\setCJKmainfont{PingFang SC}
\usepackage{listings}
\usepackage{geometry}
\usepackage{graphicx}
\usepackage{hyperref}
% 简化后的Rust语言代码高亮定义
\lstdefinelanguage{Rust}{
    keywords={break, const, continue, crate, else, enum, extern, false, fn, for, if, impl, in, let, loop, match, mod, move, mut, pub, ref, return, self, static, struct, super, trait, true, type, unsafe, use, where, while, async, await, dyn},
    sensitive=true,
    comment=[l]{//},
    morecomment=[s]{/*}{*/},
    morestring=[b]{"},
}

\geometry{a4paper, margin=1in}
\title{Rust 基础 PDF 阅读器项目报告}
\author{Geoffrey Wang}
\date{\today}

\begin{document}

\maketitle

\section{简介}
本项目展示了如何使用 Rust 和 Actix-web 框架创建一个基础的在线 PDF 阅读器。该阅读器允许用户在浏览器中内嵌查看 PDF 文件,而不是直接下载。

\section{项目结构}
项目的基本结构如下:
\begin{verbatim}
your_project/
├── Cargo.toml
├── src/
│   └── main.rs
├── static/
│   └── index.html
└── pdfs/
    └── sample.pdf
\end{verbatim}

\section{代码说明}

\subsection{main.rs}
`main.rs` 文件是项目的核心,用于配置 Actix-web 服务器,并处理 PDF 文件的在线显示。

\begin{lstlisting}[language=rust, caption=main.rs]
use actix_files::NamedFile;
use actix_web::{web, App, HttpServer, HttpRequest, Result};
use actix_web::http::header::{ContentDisposition, DispositionType};
use std::path::PathBuf;

async fn serve_pdf(req: HttpRequest) -> Result<NamedFile> {
    let filename: String = req.match_info().query("filename").parse().unwrap();
    let path: PathBuf = format!("./pdfs/{}", filename).parse().unwrap();

    let file = NamedFile::open(path)?
        .set_content_disposition(ContentDisposition {
            disposition: DispositionType::Inline,
            parameters: vec![],
        });
    
    Ok(file)
}

#[actix_web::main]
async fn main() -> std::io::Result<()> {
    HttpServer::new(|| {
        App::new()
            .route("/pdf/{filename}", web::get().to(serve_pdf))
            .service(actix_files::Files::new("/", "./static").index_file("index.html"))
    })
    .bind("127.0.0.1:8080")?
    .run()
    .await
}
\end{lstlisting}

\subsection{index.html}
`index.html` 文件用于设置 PDF 阅读器的界面,使用 `<iframe>` 标签嵌入 PDF 文件。

\begin{lstlisting}[language=html, caption=index.html]
<!DOCTYPE html>
<html lang="en">
<head>
    <meta charset="UTF-8">
    <meta name="viewport" content="width=device-width, initial-scale=1.0">
    <title>PDF Reader</title>
    <style>
        body {
            font-family: Arial, sans-serif;
            margin: 0;
            padding: 0;
            display: flex;
            flex-direction: column;
            align-items: center;
            justify-content: center;
            height: 100vh;
        }

        h1 {
            margin-bottom: 20px;
        }

        iframe {
            border: 1px solid #ccc;
            width: 80%;
            height: 80%;
        }
    </style>
    <script>
        // 禁用右键菜单(减少直接下载的可能性)
        document.addEventListener('contextmenu', function(e) {
            e.preventDefault();
        }, false);
    </script>
</head>
<body>
    <h1>PDF Reader</h1>
    <iframe src="/pdf/sample.pdf" width="100%" height="800px"></iframe>
</body>
</html>
\end{lstlisting}

\section{运行和验证步骤}
在项目根目录下运行以下命令以启动服务器:

\begin{verbatim}
cargo run
\end{verbatim}

然后,在浏览器中访问 \url{http://127.0.0.1:8080},您应该能看到内嵌显示的 PDF 文件。如果显示正确,项目就已经成功运行。

\end{document}